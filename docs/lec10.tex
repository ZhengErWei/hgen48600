% -----------------------------*- LaTeX -*------------------------------
\documentclass[12pt]{report}
\usepackage{scribe_hgen486}
\usepackage{tikz}
\usetikzlibrary{fit,positioning}
\begin{document}

\scribe{Chris Stamper}		% required
\lecturenumber{10}			% required, must be a number
\lecturedate{February 4}		% required, omit year
\lecturer{John Novembre} 

\maketitle

% please leave this comment 
\framebox[.95\textwidth]{\parbox{.93\textwidth}{ {{\bf Note:}} These
lecture notes are still rough, and have only have been mildly
proofread.  }}
\vspace*{.1in}


% feel free to delete content below this line 
% ----------------------------------------------------------------------


\section{Hidden Markov Models}

\subsection{Hidden Markov Models Introduction}
Hidden Markov Models have a variety of applications and are widley used in many disparate fields. One common application is their use
in speech recognition. (Please see Lawrence Rabiner's 'A Tutorial on Hidden Markov Models and Selected Applications in Speech Recognition')

\subsection{Uses of Hidden Markov Models}
Hidden Markov Models are perfect for an underlying Markov Model with noisy data.
\vspace{10 mm}
\newcommand{\HMM}[1][]{
	\begin{tikzpicture}[#1]
	\centering
	\tikzstyle{main}=[circle, minimum size = 10mm, thick, draw =black!80, node distance = 12mm]
	\tikzstyle{connect}=[-latex, thick]
	\node[main, fill = white!100] (Q1) [label=below:$Q_1$] { };
	\node[main] (Q2) [right=of Q1,label=below:$Q_2$] { };
	\node[main] (Q3) [right=of Q2,label=below:$Q_3$] { };
	\node[main] (Q4) [right=of Q3,label=below:$Q_4$] { };
	\node[main, fill = black!10] (O1) [above=of Q1,label=left:$O_1$] { };
	\node[main, fill = black!10] (O2) [above=of Q2,label=left:$O_2$] { };
	\node[main, fill = black!10] (O3) [above=of Q3,label=left:$O_3$] { };
	\node[main, fill = black!10] (O4) [above=of Q4,label=left:$O_4$] { };
	\path (Q1) edge [connect] (Q2)
        (Q2) edge [connect] (Q3)
        (Q3) edge [connect] (Q4)
        (Q1) edge [connect] (O1)
        (Q2) edge [connect] (O2)
        (Q3) edge [connect] (O3)
        (Q4) edge [connect] (O4);
	\end{tikzpicture}}

\HMM{}
\vspace{10 mm}

Meaning that if one conditioned on $Q_3$ the observation would only depend on it and nothing else.

\subsection{Basic Structure of Hidden Markov Models}

The basic structure of a Hidden Markov Model is:

\begin{itemize}
	\item A Transition Probability Matrix: ($A=\{a_{ij}, i=1, ...N, j=1 ...N\}$) with
	\vspace{5 mm}
	\begin{equation}
	a_ij = P(q_t+1 = S_j |q_t = S_i), \hspace {3 mm} 1 \leq i,j \leq N, 1 \leq t \leq T - 1.
	\end{equation}
	\item An Emission Probability Matrix: $B_{NXM}$ is a matrix with all $b_j(i)$
	\vspace{5 mm}
	\begin{equation}
	P(O_t = i|Q_t=j)=b_j(i), \hspace {3 mm} 1 \leq j \leq N, 1 \leq i \leq M, 1 \leq t \leq T.
	\end{equation}
	\item An Initial State Distribution: $\pi = \{\pi_i, i=1 ..., N\}$ or with
	\vspace{5 mm}
	\begin{equation}
	\pi_i = P(q_1 = S_i), \hspace {3 mm} 1 \leq i \leq N.
	\end{equation}
\end{itemize}

\vspace {5 mm}

Which means there are three main problems for Hidden Markov Models.
\begin{enumerate}
 \item $P(O|\lambda)$: How do we compute efficiently?
 \item Given $O$ and $\lambda$, what is the most probable sequence, $Q$.
 \item How can we estimate $\lambda$? AKA How can we find a $rp_{max} \lambda P(O|\lambda)$?
\end{enumerate}

\section{Algorithms for solving the problmes of Hidden Markov Models}

\subsection{Forward Algorithm}

For problem 1 of HMMs we need a way to solve
\[P(O|\lambda) = \sum_Q P(O, Q|\lambda)\]

\vspace {5 mm}

Which we can accomplish by using the forward algorithm. 
\begin{itemize}
	\item Initialize
	\begin{equation}
	\alpha_1(i) = \pi_ib_i(O_1) = P (O, Q_1)
	\end{equation} 
	\item Induction
	\begin{equation}
	\alpha_{t+1}(j) = [\sum_{i=1}^N \alpha_t(i)a_{ij}]b_j(O_{t+1}), \hspace{3 mm} 1 \leq t \leq T-1 \hspace{2 mm} and \hspace{2 mm} 1 \leq j \leq N
	\end{equation}
	\item Termination
	\begin{equation}
	P(O|\lambda) = \sum_j\alpha_T(j)	
	\end{equation}
\end{itemize}

\subsection{Viterbi Algorithm}

For Problem 2 we need a way for the argmax of $P(O,Q|\lambda)$. In other words we need a way to consider all possible $Q$ and find the $Q$ that is maximal.

\newcommand{\latice}[2][]{
	\begin{tikzpicture}[#2]
	\centering
	\tikzstyle{main}=[circle, minimum size = 6mm, node distance = 5mm]
	\tikzstyle{connect}=[-latex, thin]
	\node[main, fill = black!100] (1) [label=left:$1$] { };
	\node[main, fill = black!100] (2) [below=of 1,label=left:$2$] { };
	\node[main, fill = black!100] (3) [below=of 2,label=left:$3$, label=below:$t_1$] { };
	\node[main, fill = black!100] (4) [right=of 1] { };
	\node[main, fill = black!100] (5) [below=of 4] { };
	\node[main, fill = black!100] (6) [below=of 5, label=below:$t_2$] { };
	\node[main, fill = black!100] (7) [right=of 4] { };
	\node[main, fill = black!100] (8) [below=of 7] { };
	\node[main, fill = black!100] (9) [below=of 8, label=below:$t_3$] { };
	\path (1) edge [connect] (6);
	\path (2) edge [connect] (6);
	\path (3) edge [connect] (6);
	\path (4) edge [connect] (7);
	\path (4) edge [connect] (9);
	\path (4) edge [connect] (9);
	\path (5) edge [connect] (7);
	\path (5) edge [connect] (8);
	\path (5) edge [connect] (9);
	\path (6) edge [connect] (7);
	\path (6) edge [connect] (8);
	\path (6) edge [connect] (9);
\end{tikzpicture}}

\vspace{2 mm}
To visualize this we can imagine a lattice graph:
\newline
\vspace{5 mm}
\latice{}
\vspace {1 mm}\\
In which we need to take the maximum of the three paths coming from some $i$ at time 2 and going to some $j$ at time 3. \\

To do this we can make use of the Viterbi Algortithm.
\\
In which we have:
\begin{itemize}
	\item An array defined as: $\psi_{t+1}j = argmax_i[\delta_t(i)a_{ij}], \hspace{3 mm} 1 \leq j \leq N, 1 \leq t \leq T-1$
	\item An auxillary variable, $\delta$: $\delta_t(i) = max_{q_1q_2...q_{t-1}}P\{q_1,q_2,...q_{t-1}=i,O_1,O_2,...,O_{t-1}|\lambda\}$
\end{itemize}
Which we can use in the algortihm.
\begin{itemize}
	\item Initialize
	\begin{equation}
	\begin{split}
	\delta_1 (i) \leftarrow \pi_ib_i(O_1), \hspace{3 mm} 1 \leq i \leq N \hspace{3 mm} \\
	\psi_1 (i) \leftarrow 0, \hspace{3 mm} 1 \leq i \leq N \\
	t \leftarrow 1
	\end{split}
	\end{equation} 
	\item Repeat
	\begin{equation}
	\begin{split}
	\psi_{t+1}(j) \leftarrow argmax_i[\delta_t(i)a_(ij)], \hspace{3 mm} 1 \leq j \leq N \\
	\delta_{t+1}(j) \leftarrow \delta_t(\psi_{t+1}(j))a_{\psi_{t+1}(j),j}b_j(O_{t+1}), \hspace{3 mm} 1 \leq j \leq N \\
	t \leftarrow t + 1
	\end{split}
	\end{equation}
	\item Until
	\begin{equation}
	\begin{split}
	t = T \\
	P^* \leftarrow max_{1 \leq i \leq N}[\delta_T(i)] \\
	i_T^* \leftarrow argmax_{1 \leq i \leq N}[\delta_T(i)] \\
	\text{State Sequence Backtracking:} \\
	q_T^* \leftarrow S_{i_T^*} \\
	t \leftarrow T
	\end{split}
	\end{equation}
	\item Repeat
	\begin{equation}
	\begin{split}
	i_{t-1}^* \leftarrow \psi_t(i_t^*) \\
	q_{t-1}^* \leftarrow S_{i_{t-1}^*} \\
	t \leftarrow t - 1 \\
	\text{until} \\
	t = 1 \\
	Q^* \leftarrow q_1^*,...q_T^*
	\end{split}
	\end{equation}
\end{itemize}

\subsection{Forward-Backward Algorithm}
To solve the final problem, efficiently learning the parameters of a HMM, we can make use of a backward recursive procedure. \\
Our goal is that we want the backward variable, $B_t(i)$, or: \\
\vspace{3 mm}
$B_t(i) = P(O_{t+1},...,O_T|Q_t=j,\lambda)$ \\
\vspace{3 mm}
such that \\ $\alpha_t(j)B_t(j) = P(O_1,...,O_T|Q_t=j,\lambda)$
$\propto P(Q_t=j|O,\lambda)$ \\
\vspace{3 mm}
and \\
\vspace{3 mm}
$P(Q_t=j|O,\lambda) = \frac{\alpha_t(j)B_t(j)}{\sum_i\alpha_t(i)B_t(i)} = \lambda_t(j)$ \\

To compute this we can use the Backwards Algorithm, AKA the Forward-Backward Algorithm:

\begin{itemize}
	\item Initialize
	\begin{equation}
	B_T(i)=1
	\end{equation}
	\item Induction
	\begin{equation}
	B_t(i) = \sum_{j=1}^Na_{ij}b_j(O_t)B_t(j), \hspace{3 mm} t=T-1, T-2,..., 1 \hspace{1 mm} \text{and} \hspace{1 mm} 1 \leq i \leq N
	\end{equation}
	\item Continue unitl
	\begin{equation}
	P(Q_t=j|O,\lambda) = \frac{\alpha_t(j)B_t(j)}{\sum_i\alpha_t(i)B_t(i)} = \lambda_t(j)
	\end{equation}
\end{itemize}
\end{document}